\documentclass[a4paper]{report}
\usepackage{apacite}
\usepackage{graphicx}
\graphicspath{{Images/}}


\begin{document}
	%------------------------Cover-------------------------------------------------------------
	\begin{titlepage} 
		\newcommand{\HRule}{\rule{\linewidth}{0.5mm}}
		
		\center 
		
		\textsc{\large Project block 1.3}\\[0.5cm] 
		
		\HRule\\[0.4cm]
		
		{\huge\bfseries Find the smallest range for the chromatic number of a graph}\\[0.4cm] 
		
		\HRule\\[1.5cm]
		
		\textsc{\large Group 10}\\[0.5cm]

		\begin{minipage}{0.6\textwidth}
			\begin{flushleft}
				Tu Anh Dinh\\Michal Jarski\\Louis Mottet
			\end{flushleft}
		\end{minipage}
		~
		\begin{minipage}{0.3\textwidth}
			\begin{flushleft}
				Vaishnavi Velaga\\Rudy Wessels\\Oskar Wielgos
			\end{flushleft}
		\end{minipage}
		
		\vspace{2cm}
		
		Summited: Wednesday January 23, 2019
		
		
	\end{titlepage}
	
	
	
	
	
	%-------------------Title page-----------------------------------------------------------
	\begin{titlepage} 
		\newcommand{\HRule}{\rule{\linewidth}{0.5mm}} 
		
		\center
		
		\textsc{\LARGE Maastricht University}\\[1.5cm]
		
		\textsc{\Large Department of Data Science and Knowledge Engineering}\\[0.5cm] 
		
		\textsc{\large Project block 1.3}\\[0.5cm] 
		
		\HRule\\[0.4cm]
		
		{\huge\bfseries Find the smallest range for the chromatic number of a graph}\\[0.4cm] 
		
		\HRule\\[1.5cm]
		
		\textsc{\large Group 10}\\[0.5cm]
		
		\begin{minipage}{0.6\textwidth}
			\begin{flushleft}
				Tu Anh Dinh\\Michal Jarski\\Louis Mottet
			\end{flushleft}
		\end{minipage}
		~
		\begin{minipage}{0.3\textwidth}
			\begin{flushleft}
				Vaishnavi Velaga\\Rudy Wessels\\Oskar Wielgos
			\end{flushleft}
		\end{minipage}
	
		 \vspace{1cm}
		Summited: Wednesday January 23, 2019
		\vspace{3cm}
		\begin{flushleft}
			Project coordinator: Prof. Jan Paredis
		\end{flushleft}
		
	\end{titlepage}
	
	%-----------------------------------------------------------------------------
	\chapter*{Preface}
	\pagenumbering{gobble}

	\addcontentsline{toc}{chapter}{Preface}
	

	\chapter*{Summary}
	\addcontentsline{toc}{chapter}{Summary}
	
	\tableofcontents
	
	\chapter*{Abbreviations and symbols}
	\addcontentsline{toc}{chapter}{List of abbreviations and symbols}
	
	%-----------------------------------------------------------------------------
	\chapter{Introduction}
	\pagenumbering{arabic}

	
	\chapter{Methods}
		\section{Decomposing the graph}
		One graph can contain multiple disconnected parts, which can be considered as independent graphs. This section describes the method for decomposing the graph into these subgraph.\\
		The algorithm used is based on breadth-first search. We use a list to store the vertices whose neighbors are not yet added to the same subgraph. First, add the first vertex to a subgraph and the unchecked list. Then add all neighbors of the vertex to the subgraph and the unchecked list, and remove the first vertex from the unchecked list. We do the same for all element
		Use breadth-first search\\
		Usage: Allow other methods to work on smaller graphs
		
		\section{Greedy algorithm}
		Sort the vertices based on their constraints\\
		Try to reuse available colors\\
		Usage: Find the upper-bound
		
		\section{Lower-bound}
		
		\section{Special cases}
			\subsection{Bipartite}
			Use breadth-first search\\
			Chromatic number = 2
			\subsection{Odd cycle}
			Chromatic number = 3
			\subsection{Complete graph}
			Check if every vertex is connected to all other vertices\\
			Chromatic number = number of vertices
			\subsection{Wheel graph}
			
		\section{Genetic algorithm}
			\subsection{Fitness function}
			Based on the number of invalid colorings of each graph
			\subsection{Selection method}
			\subsection{Crossover}
			\subsection{Mutation}
		
		\section{Brute force search}
		
		
	\chapter{Experiments}
		
	\chapter{Results}
	
	\chapter{Discussion}
	
	\chapter{Conclusion}
	
	
	\bibliographystyle{apacite}
	\bibliography{references}
	
	\appendix
	\chapter*{Appendix}
	\addcontentsline{toc}{chapter}{Appendix}
\end{document}